\documentclass[10pt]{beamer}

\usepackage[utf8]{inputenc}
\usepackage[brazil]{babel}
\usepackage{graphicx}		% utilizado para inserir gráfico

% Usado para incluir código
\usepackage{listings}

\usetheme{Copenhagen}
\usecolortheme{lccv}
\setbeamertemplate{background}{\includegraphics[width=\paperwidth]{./figuras/background_lccv.jpg}}
%\setbeamercovered{transparent}

\title[]{Uma introdução a programação em computação de alto desempenho usando \textbf{OpenMP} e \textbf{MPI}}
\author[]{Baltazar Tavares Vanderlei}
\date{19 de Outubro de 2011}
\institute[2011]{Laboratório de Computação Científica e Visualização - LCCV/UFAL}

\begin{document}

\newcommand{\til}{\~{}}

\frame{\titlepage}
	\begin{frame}[t]
	\frametitle{Sumário}
	\tableofcontents[framebreaks]
	%\tableofcontents[pausesections]
\end{frame}

% Perfumaria sobre o sumário ser mostrado a cada passagem de sessão e sub-sessão.
\AtBeginSection[]{
	\begin{frame}[t]
	\frametitle{Sumário}
	\tableofcontents[currentsection]
	\end{frame}
}

\AtBeginSubsection[]{
	\begin{frame}[t]
	\frametitle{Sumário}
	\tableofcontents[currentsubsection]
	\end{frame}
}

\section{O que é?}
	\begin{frame}%[t]
	\frametitle{O que é computação de alto desempenho?}
		\text Conjunto de tecnicas e tecnologias para a resolução de problemas em escala maior que o usual.
		\pause
		\begin{itemize}[<+->]
			\item Entenda como \textbf{usual} como tecnologias populares atualmente.
			\item Problemas de \textbf{C.A.D.} não são resolvido em tempo habil normalmente.
			\item Quando um problema passa a ser resolvido em tempo habil, sempre você tem um problema maior.
		\end{itemize}
	\end{frame}

	\begin{frame}%[t]
	\frametitle{Problemas ao longo do tempo...}
		\begin{figure}
		\centering
			\includegraphics[scale=0.2]{./figuras/problema_x_tempo.png}
			\caption{Problemas ao longo do tempo}
		\end{figure}
	\end{frame}

\section{Quem usa?}
	\begin{frame}
	\frametitle{Que usa a Computação de Alto Desempenho?}
		\begin{itemize}
			\item Programas que usam recurso de forma irresponsavel
			\pause
			\item Programas que usam recursos ao maximo disponivel
			\pause
			\begin{itemize}
				\item Processamento
				\item Memoria
				\item Disco
				\item Rede
				\item \textbf{Etc...}
			\end{itemize}
			\pause
			\item Programadores \textbf{HARDCORE}!
		\end{itemize}
	\end{frame}

	\begin{frame}%[t]
	\frametitle{O que vamos falar nesse curso?}
		\begin{itemize}[<-+>]
			\item Vamos falar de processamento em paralelo
			\item Aprender a paralelizar usando dois paradigmas(\textbf{OpenMP} e \textbf{MPI})
			\item Vamos aprender sobre programas em paralelo para C.A.D.
			\item Não vamos aprender alguns conceitos que vocês vão aprender em S.O. ;)
		\end{itemize}
	\end{frame}

\section{Serial x Paralelo}
	\subsection{Serial}
		\begin{frame}%[t]
		\frametitle{Como fazemos no mundo atual?}
			\begin{itemize}
				\item Somos acustumados a pensar de forma serial
				\item Fazemos programas ``seriais''
			\end{itemize}
			\begin{figure}
			\centering
				\includegraphics[scale=0.2]{./figuras/exemplo_programa_serial.png}
				\caption{Exemplo de programa serial}
			\end{figure}
		\end{frame}

		\begin{frame}%[t]
		\frametitle{Vamos fazer um grafo...}
			\begin{figure}
			\centering
				\includegraphics[scale=0.2]{./figuras/grafo_programa_serial.png}
				\caption{Grafo de ações seriais}
			\end{figure}
		\end{frame}

	\subsection{Paralelo}
		\begin{frame}%[t]
		\frametitle{Como fazemos no mundo real?}
			\begin{itemize}[<-+>]
				\item Apesar de ser facil de imaginar em serial, o mundo é paralelo
				\item O numero de processadores cresce mais que a capacidade de processamento
				\item O custo de paralelizar é menor do que adiquirir um recurso não paralelizado
				\item Hardware cada vez mais feito com arquitetura paralela
				\item Se você faz programas seriais, você esta deixando recurso ocioso
			\end{itemize}
% Seria interessante colocar um exemplo?
% Qual exemplo seria "ditatico" para aqui?
%			\begin{figure}
%			\centering
%				\includegraphics[scale=0.2]{./figuras/exemplo_programa_paralelo.png}
%				\caption{Exemplo de programa paralelo}
%			\end{figure}
		\end{frame}

		\begin{frame}%[t]
		\frametitle{Vamos fazer um grafo...}
			\begin{figure}
			\centering
				\includegraphics[scale=0.2]{./figuras/grafo_programa_paralelo.png}
				\caption{Grafo de ações paralelo}
			\end{figure}
		\end{frame}

		\begin{frame}%[t]
		\frametitle{Vamos comparar os grafos...}
			\begin{figure}
			\centering
				\includegraphics[scale=0.2]{./figuras/grafo_programa_paralelo_vs_serial.png}
			\end{figure}
		\end{frame}

\section{Speed-Up}
\section{Taxionomia de Flynn}
\section{MPI e OpenMP}

\end{document}

