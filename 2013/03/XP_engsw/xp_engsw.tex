\documentclass[10pt]{beamer}


\usepackage[utf8]{inputenc}
\usepackage[brazil]{babel}
\usepackage{graphicx}		% utilizado para inserir gráfico

% Usado para incluir código
\usepackage{listings}
% Needed to use citations.
\usepackage{cite}

\usetheme{Copenhagen}
\usecolortheme{lccv}
\setbeamertemplate{background}{\includegraphics[width=\paperwidth]{./figuras/background_lccv.jpg}}
%\setbeamercovered{transparent}

\title[]{eXtreme Programming}
\author[]{Baltazar Tavares Vanderlei\\
Dielson Sales de Carvalho\\
Priscylla Silva\\
Vinícius dos Santos Oliveira}
\date{\today}
\institute[2013]{Instituto de Computação - IC/UFAL}

\begin{document}

\newcommand{\til}{\~{}}

\frame{\titlepage}
\begin{frame}[t]
  \frametitle{Sumário}
  \tableofcontents[framebreaks]
  % \tableofcontents[pausesections]
\end{frame}

% Perfumaria sobre o sumário ser mostrado a cada passagem de sessão e sub-sessão.
\AtBeginSection[]{
  \begin{frame}[t]
    \frametitle{Sumário}
    \tableofcontents[currentsection]
  \end{frame}
}

\AtBeginSubsection[]{
  \begin{frame}[t]
    \frametitle{Sumário}
    \tableofcontents[currentsubsection]
  \end{frame}
}

\section{XP (ou eXtreme Programming)}
\begin{frame}
  \frametitle{Visão de alto nível}
  \begin{itemize}%[<+->]
  \item Boas práticas levadas ao extremo
    \begin{itemize}
    \item Envolvimento com o cliente
    \item TDD e revisão de código
    \item Entrega incremental
    \item Simplicidade
    \end{itemize}
  \end{itemize}
\end{frame}

\begin{frame}
  \frametitle{Itens polêmicos}
  \begin{itemize}
  \item Requerimentos instáveis
  \item Documentação menos completa, quando comparada a processos pesados
  \item Programar para o agora pode tornar o trabalho de amanhã mais pesado
  \item Sistemas críticos
  \end{itemize}
\end{frame}

\subsection{Atividades}
\begin{frame}
  \frametitle{Programação}
  \begin{itemize}
  \item Sem código, não há produto
  \item Pode ser usado para auxiliar a comunicação
  \end{itemize}
\end{frame}

\begin{frame}
  \frametitle{Testes}
  \begin{itemize}
  \item Testes unitários
  \item Testes de aceitação
  \item Testes de integração
  \end{itemize}
\end{frame}

\begin{frame}
  \frametitle{Escutar}
  Programadores devem escutar o que os clientes precisam que o sistema faça.

  \pause
  Programadores devem entender o sistema bem o suficiente para dar feedback
  sobre os detalhes técnicos de como o problema pode ser resolvido.
\end{frame}

\begin{frame}
  \frametitle{Projeto}
  \begin{itemize}
  \item Sistema muito complexo
  \item As dependências não ficam claras
  \end{itemize}
\end{frame}

\subsection{Valores}
\begin{frame}
  \frametitle{Comunicação}
  \begin{itemize}
  \item Arquiteturas simples
  \item Metáforas comuns
  \item Colaboração de usuários e programadores
  \item Comunicação verbal frequente
  \item Feedback
  \end{itemize}
\end{frame}

\begin{frame}
  \frametitle{Simplicidade}
  \begin{itemize}
  \item Solução mais simples primeiro
  \item Funcionalidade extra pode ser adicionada depois
  \item Relacionado ao valor de comunicação
  \end{itemize}
\end{frame}

\begin{frame}
  \frametitle{Feedback}
  \begin{itemize}
  \item Feedback do sistema
  \item Feedback do cliente
  \item Feedback do time
  \end{itemize}
\end{frame}

\begin{frame}
  \frametitle{Coragem}
\end{frame}

\begin{frame}
  \frametitle{Respeito}
\end{frame}

\end{document}
